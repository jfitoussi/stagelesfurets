\chapter{Introduction}
Dans le cadre de mon projet de fin d'études il m'a été demandé de réaliser un projet au sein de l'entreprise dans laquelle je réalisais mon alternance. Après avoir examiné l'ensemble des chantiers techniques en attente dans l'entreprise, je me suis trouvée nez à nez avec "Modélisation du graphe des champs des formulaires". L'architecture de l'application Web ainsi que le business de LesFurets.com est centré autour des \textbf{formulaires permettant aux internautes de cibler leur profil.} En effet le site LesFurets.com est un comparateur d'assurance (Auto, Moto, Habitation, Crédit et Mutuelle Santé), de crédit à la consommation et de forfaits mobile. Avec LesFurets.com, les utilisateurs peuvent comparer facilement les offres d’un panel d’assureurs pour trouver le produit qui convient le mieux. D’un point de vue technique l’équipe desFurets.com a décidé d’\textbf{abstraire les champs des formulaires ainsi que leurs dépendances à l’aide d'un metamodel}. A l’heure actuelle, l’application web est capable de transformer ce modèle de champs d'un formulaire au format XMI pour pouvoir les exporter dans un logiciel de modélisation comme MagicDraw, Rational. L'objet de mon projet sera de proposer un \textbf{outil capable d'afficher sous forme de graphe ce metamodel} ce modèle. Pour mener a bien le projet il me faudra, dans un premier temps, passer à d’autres format de modélisation (EMF, JSON) qui pourront être lus à l'aide d'outils open-source. Il est ensuite prévu de superposer des trackers analytics sur ces graphes. Il est aussi envisagé de garder une trace des différences entre les versions successives de l’application pour visualiser les modifications et/ou les régressions en fonctions des développements. Enfin au même titre qu'un IDE classique, l’outil devrait pouvoir proposer des fonctions d’édition, de sauvegarde et de partage des données traitées.