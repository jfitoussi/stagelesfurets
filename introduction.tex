\chapter{Introduction}
Dans le cadre de mon projet de fin d'études il m'a été demandée de réaliser un projet dont je participer à la conception d'un logiciel au seins de l'entreprise dans laquelle je réalisais mon alternance. Après avoir examiner l'ensemble des chantiers techniques en attente dans l'entreprise, je me suis trouvée nez à nez avec "Modélisation du graphe des champs des formulaires". L'architecture de l'application Web ainsi que le business de LesFurets.com tournent énormément autour des formulaires permettant aux internautes de cibler leur profil. En effet le site LesFurets.com. LesFurets.com est un comparateur d'assurance (Auto, Moto,Habitation, Crédit et Mutuelle Santé), de crédit à la consommation et de forfaits mobile. Avec LesFurets.com, les internautes peuvent comparer facilement les offres d’un panel d’assureurs pour trouver le produit qui leurs convient. Pour cela le site propose des formulaires permettant de cibler les besoins des futures assurées. D’un point de vue technique l’équipe de lesfurets.com a décider d’abstraire les champs du formulaire ainsi que les dépendances à l’aide d'un méta-model. A l’heure actuelle, l’application web est capable de transformer ce modèle de champs au format XMI pour pouvoir les exporter dans un logiciel de modélisation comme MagicDraw, Rational. Le stage débutera par un passage à d’autres format de modélisation (EMF, JSON) traiter à l'aide d'outils open-source qui permettront aux équipes de mieux visualiser le graphe des champs des formulaires ainsi que leurs dépendances. Il est prévu ensuite de superposer des analytics sur ces graphes. Il est aussi envisager de garder une trace des différences entre les versions de l’application pour visualiser les modifications et/ou les régressions en fonctions des développements. Enfin au même titre qu’un IDE classique, l’outil devrai pouvoir proposer des fonctions d’édition, de sauvegarde et de partage des données traitées.
