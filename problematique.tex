\chapter{Problématique}

\section{Contexte}
L'application du site lesfurets.com est construite sur une architecture JAVA SE packager a l’aide de l'outils de gestion et d'automatisation de production des projets logiciels Maven. Maven s’occupe de gerer les dépendances de code générer certaines classes ainsi que de compiler le projet. Chaque formulaire est représenté au seins du site par un tunnel, toute la logique est est gerer par GWT (Google Web Toolkit) qui s’occupe de générer du code javascript depuis du code java. L'intérêt du framework est de permettre aux développeur java de l’entreprise d’aborder leur code coté client et coté serveur avec la même approche. De plus GWT met l'accent sur des solutions efficaces et réutilisables aux problèmes rencontrés habituellement par le développement AJAX : difficulté du déboggage JavaScript, gestion des appels asynchrones, problèmes de compatibilité entre navigateurs, gestion de l'historique et des favoris, etc.
Au seins de l’application il existe déjà un projet pour générer des fichiers mdxml et csv représentant les champs du formulaire c’est dans cette partie du code que l’on développera l’outils de modélisation.
Pour la modélisation en elle même Il s'agira essentiellement d'identifier les entités logiques et les dépendances logiques entre ces entités. La modélisation des données est une représentation abstraite, dans le sens où les valeurs des données individuelles observées sont ignorées au profit de la structure, des relations, des noms et des formats des données pertinentes.
Pour simplifié la problématique, l’équipe de lesfurets.com a décider d’abstraire les champs du formulaire ainsi que les dépendances à l’aide d'un méta model. Un méta modèle décrit la structure des modèles et permet de raisonner sur les modèles comme sur les connaissances de premier niveau.

\section{Besoins}
Du coté des équipes les besoins était double dans un premier lieu donnée une visibilité plus forte a la mécanique déployer pour les utilisateurs au seins des tunnels. Mettre en relations les données récolté sur les parcours utilisateurs et le parcours utilisateurs lui même. Et d’autre part la visualisation des modifications possible du graph lui même
