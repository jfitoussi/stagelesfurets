\chapter{Conclusion}

Au terme de cette deuxième période d'apprentissage, j'ai pu parfaire mon expérience chez LesFurets.com. Au sein d'une équipe motivée et dynamique j'ai réussi à m'intégrer au groupe, tout en intégrant les méthodologies utilisées. J'ai essentiellement travaillé sur mon projet de fin d'études, mais j'ai aussi pu travailler sur divers sujets en cours. J'ai pu développer de nouvelles fonctionnalités  et j'ai pu les mettre en production. J'ai pu assister à a des réunions, des points techniques, des conférences et des formations, m'enrichir en débattant autour de relecture du code de mes collègues et aussi de leur relecture du mien. Mon sujet de stage m'a permis d'élaborer une solution technique à un problème qui concernait des gens habitués à être confronté à des développeurs. J'ai pu être challengé aussi bien d'un point de vue technique que fonctionnel. Aujourd'hui l'outil que j'ai développé se trouve dans les pages de maintenance de l'entreprise et peut être utilisé par celui qui en a besoin. De plus il est amené à évoluer au fil du temps avec l'application.

\section*{Remerciements}
\phantomsection
\addcontentsline{toc}{section}{Remerciements}
J'adresse mes remerciements aux personnes qui m'ont aidé dans la réalisation de ce projet.
Tout d'abord l'équipe pédagogique du master STL/INSTA de l'UPMC. M. Emmanuel Chailloux, M. Binh-Minh Bui-Xuan et M Philippe Trebuchet pour nous avoir suivies toute l'année avec bienveillance et pédagogie.
Toutes les personnes de l'entreprise Courtanet qui m'ont aider dans la réalisation de l'outil et avec qui j'ai passer de très bon moments tout au long de l'alternance, Benjamin Degerbaix, Gilles Di Guglielmo, toute l'équipe "Traffic", l'equipe "Journey" ainsi que tout le département IT de l'entreprise.
Enfin je souhaiterais adresser des remerciement à tous les étudiants de la promo STL/INSTA 2015 avec qui nous avons construit un grand esprit de camaraderie, d'écoute et d'entraide.