\chapter{LesFurets.com}

\section{L'Assurance}
Le secteur de l’assurance fait partie du secteur économique tertiaire qui regroupe les
industries de services essentiellement immatériel. Il s’agit d’un secteur d’activité où les
acteurs économiques vendent une protection contre le risque avec un poids très important
au sein du paysage économique mondial et, plus particulièrement, au sein du paysage
économique français. Sa particularité réside dans son activité et sa diversité,puisque il
rassemble une multitude d’acteurs : compagnies d’assurance, courtiers, mutuelles, acteurs
internet, etc. Mondialement, le marché de l’assurance est actuellement dominé par les
états-Unis suivi par le Japon. La France occupe la 4ème position, derrière le Royaume Uni,
avec 6,5\% des cotisations mondiales, soit près de 211 235 millions d’euros. (FFSA
et Gema)

\section{Les comparateurs d'Assurance}
Un comparateur de prime d’assurance est un outil informatique permettant de comparer
toutes les offres d’assurances répondant à un ensemble de critères mentionnés par
le client. Avec un marché fortement concurrentiel et très complexe, les comparateurs ont
un rôle de plus en plus important auprès des clients en matière de comparaison, de choix
et de souscription de produits d’assurance. Aujourd’hui, le marché est occupé par des
concurrents connus tels que LeLynx.fr, LesFurets.com, Assurland.com, Hyperassur, automotocompare.fr,
kelassur.com, etc. En effet, ces comparateurs présentent des environnements
ergonomiques permettant aux visiteurs d’effectuer un comparatif de contrats auto,
moto, multirisques habitation, emprunteur ou santé, et de sélectionner les offres les mieux
adaptées à leurs attentes. Pour répondre aux besoins des clients, chaque acteur maintient son propre modèle qui repose sur plusieurs aspects : le tarif, la qualité des services proposés,
les services après-vente, les garanties, les franchises, les prestations, etc. Ainsi, chaque
comparateur présente des offres d’assureurs différents. Le référencement web et la publicité
présentent des investissements lourds pour l’ensemble des concurrents qui cherchent
à obtenir une visibilité conséquente auprès de leur clients. Les nouveaux arrivants doivent
forcément investir massivement pour trouver une place parmi les acteurs historiques sur
ce marché.

\section{Historique de l'Entreprise}
Courtanet, fondée en 2005 par Jehan de Castet, est la société éditrice du site AssureMieux.com.
Avec ses sites www.assuremieux.com, www.creditmieux.com, et son logiciel
Bénéfit, la société est devenue rapidement l’un des plus importants fournisseurs
français de solutions de comparaison d’assurances. Plus de 1 500 courtiers indépendants
français étaient équipés de ses solutions. Depuis Novembre 2010, Courtanet est majoritairement
détenu par BGL Group créé en 1992, intermédiaire britannique d’assurances et
propriétaire également du leader de la comparaison d’assurances en Grande Bretagne via
www.comparethemarket.com, qui a emporté 35% du marché au Royaume-Uni. Le groupe
est aujourd’hui structuré autour de quatre «piliers» : Entreprises intermédiaires, Entreprises
pilotées par marque (LesFurets.com), comparethemarket.com et services juridiques.
En Avril 2012, la plateforme de comparaison Assuremieux change d’identité et dé-
voile une nouvelle marque baptisée www.LesFurets.com, le comparateur qui « simpli-
fie » l’assurance. LesFurets.com est un comparateur indépendant et impartial d’assurance
auto, moto, santé, habitation et crédit. Sa mission est d’aider les consommateurs
à trouver l’assurance la mieux adaptée à leurs besoins au meilleur tarif. La mission
du site est d’apporter plus de transparence et de simplicité dans l’assurance. Ainsi,
il permet de comparer facilement les tarifs, les garanties, les franchises et les services
de grands assureurs auto, santé,moto, habitation et crédit. Son objectif est d’aider à
trouver l’assurance la mieux adaptée aux besoins des clients au meilleur tarif. Dans
le but d’offrir un service de qualité, LesFurets.com s’associe avec les plus grands assureurs
du marché : Direct Assurance, Amaguiz, IdMacif, Eurofil, AllSecur, Aon Asurances,
AcommeAssure, SOS Malus, Assurpeople, L’olivier Assurances, 4assur, EuroAssurance,
ActiveAssurances, AssurOnline, assurbike, aloa Assurances, SwissLife, Alptis, etc.