\chapter{Gestion de projet}

Aprés une semaine passé à essayer de mettre en place une solution technique élégante, il est apparu évident pour mon tuteur que nous devions definir une methodologie pour construire l'outils de modélisation. Au seins de l'equipe IT de LesFurets.com, une methodologie Agile type Kanban à été mise en place.

\section{Agile}
Les méthodes agiles sont des groupes de pratiques de projets de développement en informatique (conception de logiciel), pouvant s'appliquer à divers types de projets. Elles ont pour dénominateur commun l'Agile manifesto. Rédigé en 2001, celui-ci consacre le terme d'« agile » pour référencer de multiples méthodes existantes. Les méthodes agiles se veulent plus pragmatiques que les méthodes traditionnelles. Elles impliquent au maximum le demandeur (client) et permettent une grande réactivité à ses demandes. Elles visent la satisfaction réelle du client en priorité aux termes d'un contrat de développement.
Les méthodes agiles reposent sur une structure (cycle de développement) commune (itérative, incrémentale et adaptative), quatre valeurs communes déclinées en douze principes communs desquels découlent une base de pratiques, soit communes, soit complémentaires.
Les méthodes pouvant se qualifier d'agile depuis la publication de l'agile manifesto sont le RAD (développement rapide d'applications) (1991) avec DSDM, la version anglaise du RAD (1995) ainsi que plusieurs autres méthodes, comme ASD ou FDD qui reconnaissent leur parenté directe avec RAD. Les deux méthodes agiles désormais les plus utilisées sont : la méthode Scrum qui fut présentée en 1995 par Ken Schwaber puis publiée en 2001 par lui même et Mike Beedle pour enfin être diffusée mondialement par Jeff Sutherland ainsi que la méthode XP Extreme programming publiée en 1999 par Kent Beck.
Un mouvement plus large (management agile) couple les valeurs agiles aux techniques de l'amélioration continue de la qualité (plus particulièrement le Lean). On constate un élargissement de l'utilisation d'agile à l'ensemble de la structure de l'entreprise.

\section{Kanban}
Parmis les methodes Agile il existe le Lean Kanban.
Kanban est un terme japonais signifiant « fiche » ou « étiquette ».
Cette méthode a été initialement mise en place dans les usines Toyota fin dans les années 60.
Parmi les outils agiles, Scrum et Kanban sont aujourd'hui les plus utilisés dans le cadre de la réalisation de projets informatiques.Ce sont tous deux des outils de processus mais néanmoins bien différents.Il est à noter que Kanban est plus adaptatif que Scrum qui fixe un cadre de travail plus rigide.

L'approche Kanban consiste globalement à visualiser le Workflow (Le processus de traitement d'une tâche). On met en place un tableau de bord des items (demandes). Chaque item est placé à un instant donné dans un état. L'item évolue jusqu'à ce qu'il soit soldé. Chaque état du tableau peut contenir un nombre maximum prédéfini de tâches simultanées (défini selon les capacités de l'équipe) : on limite ainsi le WIP (Work In Progress). Il est primordial, pendant l'exécution des tâches, de mesurer le "lead-time". Il s'agit du temps moyen pour compléter un item. Cette durée sera progressivement de plus en plus courte et prévisible.

Les intérêts de la mise en place de cet outil Kanban sont principalement :
\begin{itemize}
\item 
Possibilité de mise en place progressive de la méthodologie Agile (moins directif que Scrum)
Les points de blocages sont visibles très tôt. La collaboration dans l'équipe est encouragée pour résoudre les problèmes de manière corrective, le plus tôt possible.
\item
On peut se passer de la notion de sprint. Un sprint d'une ou deux semaines n'est pas envisageable dans certains cas de figure (réactivité supérieure exigée). La méthodologie Kanban est donc utilisée dans des services de support au client (gestion des tickets d'incidents).
\item
Facilité de communication sur l'état d'avancement du projet.
\item 
La Définition du Done (Ensemble de critères permettant de considérer la tâche comme traitée) permet de garantir un niveau de qualité constant et défini de manière collective.
\end{itemize}

\section{Kanban chez lesFurets}
Depuis 3 ans lesFurets utilisent la méthodologie Kanban.
Toutes les taches sont gérer depuis JIRA, un logiciel disponible depuis un client web. JIRA permet la création de tickets qui correspondent a des taches. Ces taches sont attribuées aux personnels de l'IT. Les ticket changent d'états et passent par une phase d'analyse, une phase de developpement, une phase de test et enfin par la phase de production. Le processus décrit dans le tableau Kanban est automatisé pour les tickets et une vue sous forme de tableau Kanban est disponible depuis le navigateur web. Cependant il existe un tableau physique représentant les taches et leur état pour respecter le concept de management visuel.

\section{Git}
Ma plus grande difficulté à été de comprendre le gestionnaire de version qui était mise en place chez LesFurets. Issue du monde de SVN, je ne comprenais pas les mots tel que pull, push et même le comportement d'un commit me paraissait compliqué a prendre en main. C'est par la suite que je compris que plus qu'un simple gestionnaire de version comme je l'utilisais autrefois Git était une pierre angulaire chez LesFurets de la gestion du projet. Git leur servait a la fois à tester le code, à faire avancer les développement en cours et comme je l'expliquerai dans la partie sur Feature Branch la détection et la résolution de conflit.

\section{Intégration continu}
LesFurets adopte de plus l'intégration continu. L'intégration continue est un ensemble de pratiques utilisées en génie logiciel consistant à vérifier à chaque modification de code source que le résultat des modifications ne produit pas de régression dans l'application développée. Le concept a pour la première fois été mentionné par Grady Booch et se réfère généralement à la pratique de l'extreme programming. Le principal but de cette pratique est de détecter les problèmes d'intégration au plus tôt lors du développement. De plus, elle permet d'automatiser l'exécution des suites de tests et de voir l'évolution du développement du logiciel.
L'intégration continue est de plus en plus utilisée en entreprise afin d'améliorer la qualité du code et du produit final. L'interet de cette pratique est que à chaque changement du code, l'application va exécuter un ensemble de tâches et produire un ensemble de résultats, que le développeur peut par la suite consulter. Cette intégration permet ainsi de ne pas oublier d'éléments lors de la mise en production et donc ainsi améliorer la qualité du produit
C'est ainsi qu'a chaque fois qu'un développer termine une fonctionnalité, les modifications doivent passer la batteries de test complète. Le logiciel utilisé pour gérer l'intégration est TeamCity il est toutefois couplé a Jenkins.

\section{Deploiement continu}
De plus lesFurets ont aussi expérimenter la culture DevOps. Devops est un mouvement visant à réduire la friction organisationnelle entre les "devs" (chargés de faire évoluer le système d'information) et les "ops" (chargés d'exploiter les applications existantes).
Ce que l'on pourrait résumer en travailler ensemble pour produire de la valeur pour l'entreprise. Dans la majorité des entreprises, la valeur sera économique mais pour d'autres elle sera sociale ou morale. C'est ainsi que les mises en production sont réalisées par les développeurs. Il m'est arriver à plusieurs reprises de mettre en productions des fonctionnalités que j'avais développé.

\section{Feature Branch}
Enfin pour assurer une développement optimale de chaque fonctionnalité la méthodologie utilisé est Feature Branch. Chaque nouvelle fonctionnalité est développé à partir d'une version du master. Une fois le développement terminée la branche doit passer par un mécanisme d'"octopus" qui permet de détecter des conflits présent avec d'autre développement en cours. Ensuite par un dialogue entre les branches en conflit, on peut résoudre par diffèrent moyens le conflit et faire passer la fonctionnalité dans le master. On peut toutefois décider de développer une fonctionnalité qui ne sera pas tout de suite intégrer dans le master et la déposer toutefois sur le repository distant. Il suffira de préciser que celle-ci n'est pas à prendre en compte par l'octopus.

\section{Ma methodologie}
En vue de toute l'architecture mise en place chez lesFurets il m'a fallu m'adapter pour pouvoir developper l'outil tout en respectant l'eco systeme mise en place et pouvoir profiter des avantages produits par le workflow mise en place.

Tout d'abord lors d'un BBL (une conference pendant la pause dejeuner) j'ai entendu parler du Kanban "Perso" qui etait exempt de certaines regles du Kanban pour des projets non collaboratif. Par la suite l'équipe m'a pris en main pour poser les bases de mon workflow. 
C'est ainsi que sur un tableau kanban personnel (informatiser à l'aide de Trello) je me suis imposer les états suivant : \\
\begin{center}
\begin{tabular}{|c|c|c|c|}
\hline
 Backlog & To Do & Doing & Done\\
\end{tabular} \\
\end{center}
J'ai du aussi effectuer un découpage précis des taches que j'allais traiter. J'ai compléter ce tableau à l'aide de macro tache dans JIRA. Cela m'a permis de faire test les tickets sur lesquels j'étais et de prévenir mon équipe de l'avance de mon projet. Mon projet s'intégrant dans l'application il m'a fallu dans un premier développer mes fonctionnalité hors de l'octopus et régler les conflit plus tard lorsque l'outil pouvait commencer à être tester.