\chapter{Sujet de stage}
D’un point de vue technique l’équipe de lesfurets.com a décider d’abstraire les champs du formulaire ainsi que les dépendances à l’aide d'un méta-model. A l’heure actuelle, l’application web est capable de transformer ce modèle de champs au format XMI pour pouvoir les exporter dans un logiciel de modélisation comme MagicDraw, Rational. Le stage débutera par un passage à d’autres format de modélisation (EMF, JSON) traiter à l'aide d'outils open-source qui permettront aux équipes de mieux visualiser le graphe des champs des formulaires ainsi que leurs dépendances. Il est prévu ensuite de superposer des analytics sur ces graphes. Il est aussi envisager de garder une trace des différences entre les versions de l’application pour visualiser les modifications et/ou les régressions en fonctions des développements. Enfin au même titre qu’un IDE classique, l’outil devrai pouvoir proposer des fonctions d’édition, de sauvegarde et de partage des données traitées.

\section{Technologie}
Dans le cadre du projet, il me faudra explorer plusieurs piste :
La première consistera à me familiariser avec un outils de modeling basé sur Eclipse : Sirius (https://eclipse.org/sirius/overview.html). Sirius est un projet Open Source de la Fondation Eclipse. Cette technologie permet de concevoir un atelier de modélisation graphique sur-mesure en s'appuyant sur les technologies Eclipse Modeling, en particulier EMF et GMF. L'atelier de modélisation créé est composé d'un ensemble d'éditeurs Eclipse (diagrammes, tables et arbres) qui permettent aux utilisateurs de créer, éditer et visualiser des modèles EMF. Le projet est une initiative française et la communauté autour du projet est très active. Je me mettrais en relations avec les créateurs du projet pour pouvoir aborder les questions que j’aurais tout au long de mon stage.
Il me faudra aussi explorer la piste de la création une application web en charge d’afficher la modélisation. Il existe de multiple librairies en JavaScript pour l’affichage des données sous la forme de graphe orienté. De plus le il s'agit d'un graphe orienté faiblement cyclique qu'on pourra traitées, coté serveur avec des algorithmes implémenté en Java, Scala, Python ou JavaScript. Enfin aujourd’hui le graphe est généré sur un fichier au format XMI mais on pourrai imaginer que l’outil introspecte les objets directement dans les binaires.

\section{Objectif}
Les utilisateurs de l’outil seront d’une part les architectes logiciel de la société, les développeurs mais aussi les business-analyste (concepteur fonctionnelle) qui pourront mieux cerner les impacts des modifications ainsi que les besoin des utilisateurs du site lesfurets.com. Ainsi il faudra imaginer une interface qui pourrait s'apparenter à des wireframes qui ne sera plus l’apanage des seuls ingénieurs.
Fonctionnalités futures
Une fois l’outil développé et adapté aux besoin des formulaires, les équipes souhaiterait aussi modéliser l’arbre de dépendances des pages du site entre elle et de les liée aux donnée SEO déjà présentent.
